\section{Постановка задачи}

В домашней работе №1 численными методами решается задача теплопроводности вида:

\begin{array}{l}
\dfrac{\partial u}{\partial t} = \alpha \dfrac{\partial^2 u}{\partial x^2} + f(x, t), \; t > 0, \; x \in (0, l_x), \\
\left. u \right|_{t=0} = u_0(x), \\
\left. u \right|_{x=0} = \mu_1(t), \\
\left. u \right|_{x=l_x} = \mu_2(t),
\end{array}

где $u(x, t)$ - функция температуры, $\alpha = const$ - коэффициент теплопроводности, $f(x, t)$ - функция источника. 

Задача решается двумя разностными методами - явным и неявным. 
Зададим дискретную по времени (с шагом $\tau$) и пространству (с шагом $h$) сетку состояний $u(x_i,t_k) = u(h*i, \tau*k) = u^k_i$
Так же в рамках задачи оценивается шкалирование ошибки обоих методов с реальным (аналитическим решением) относительно временного шага $\tau$ и пространственного $h$.

Эксперименты проводятся на следующих двух задачах:

\begin{equation}
\begin{cases}
\dfrac{\partial u}{\partial t} = \dfrac{\partial^2 u}{\partial x^2}, \; t > 0, \; x \in (0, 1), \\
\left. u \right|_{t=0} = 0, \\
\left. u \right|_{x=0} = 1, \\
\left. u \right|_{x=l_x} = 1,
\end{cases}
\end{equation}


\begin{equation}
\begin{cases}
\dfrac{\partial u}{\partial t} = \dfrac{\partial^2 u}{\partial x^2}, \; t > 0, \; x \in (0, \pi), \\
\left. u \right|_{t=0} = \sin{(4\pi)}, \\
\left. u \right|_{x=0} = 0, \\
\left. u \right|_{x=l_x} = 0,
\end{cases}
\end{equation}

Очевидно, что в отсутствии собственного источника ($f(t,x) = 0$) и константных краевых условиях, системы имеют предельное состояние:

\[\exists \lim\limits_{t \rightarrow \infty} u(x, t) = u_\infty (x).\]

В первом случае это - $u_{\infty1} (x) = 1$, во втором - $u_{\infty1} (x) = 0$. 
Поэтому обе схемы будут считаться до установления сходимости их состояний,
то есть пока вектор разницы состояний в смежные моменты времени не будет по модулю меньше заданного пользователем $\epsilon > 0$:

\[ ||u^{k+1} - u^k|| < \epsilon \]

\section{Явный метод}:

