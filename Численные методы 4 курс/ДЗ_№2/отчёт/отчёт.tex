\section{Постановка задачи}

В домашней работе №2 численными методами решается задача Дирихле для уравнения Пуассона следующего вида:

\begin{equation}
\begin{array}{l}
\Delta{u} = 4, \; x \in (0;1), \; y \in (0;1), \\
\left.u\right|_{x = 0} = y^2, \\ \left.u\right|_{x = 1} = 1 + y^2, \\
\left.u\right|_{y = 0} = x^2, \; \left.u\right|_{y = 1} = 1 + x^2. \\
\label{eq:task}
\end{array}
\end{equation}


Задача решается двумя методами - итеративным методом Зейделя и методом Фурье. 
В обоих случаях применяется пятиточечная разностная схема вида "крест": три точки вдоль оси $x$ и три точки вдоль оси $y$:

\[ \frac{y_{i + 1, j} - 2y_{i, j} + y_{i - 1, j}}{h^2} + \frac{y_{i, j + 1} - 2y_{i, j} + y_{i, j - 1}}{h^2} = 4 \]
Точки будут браться с одинаковым расстоянием между друг другом - $h$, поэтому можно воспользоваться дискретными индексами $i,j$, где:
\[ y(i,j) = y(x_i, y_j) \]

Граничные условия для сеточной функции $y$ примут вид:
\[ y_{0, j} = y_j^2,\ \ y_{N, j} = 1 + y_j^2 \]
\[y_{i, 0} = x_i^2,\;y_{i, N} = 1 + x_i^2 \]

Известно так же точное решение задачи:

\[ u_{true} =  x^2 + y^2\]

График точного решения имеет вид:

\begin{figure}[h]
\centering
\includegraphics[width=0.7\textwidth]{True.png}
\caption{График аналитического решения задачи \ref{eq:task}}
\end{figure}

\newpage
\section{Метод Зейделя}

С одной стороны, расчёт точного решения подразумевает решения системы неявным методом а учитывая структуру разностной схемы, требуется решить их для всех переменных сразу - то есть, $O(L^2)$ уравнений системы. Итеративный метод позволяет задать некоторое состояние системы (скажем, нули везде кроме краёв) и считать значения текущей итерации по значениям предыдущей итерации.

Метод Зейделя имеет более совершенный вид:

\[ \frac{y_{i + 1, j}^n - 2y_{i, j}^{n+1} + y_{i - 1, j}^{n+1}}{h^2} + \frac{y_{i, j + 1}^{n} - 2y_{i, j}^{n+1} + y_{i, j - 1}^{n+1}}{h^2} = 4 \]

Считая в сторону возрастания индексом, метод использует точки меньших индексов, уже подсчитанных на этой итерации, а повышенных индексов - из прошлой итерации. Итерации будут продолжаться до сходимости нормы разницы между матрицами по формуле:

\[ ||y^{n+1} - y^{n}|| = \sqrt{\sum_{i=0}^{N}\sum_{j=0}^{N}(y^{n+1}_{i,j} - y^{n}_{i,j})^2 * h^2} < \epsilon  \] 

Код реализации представлен в приложении \ref{seid} - здесь за значения $y^{n}$ отвечает $u_0$, а $u$ - за $y^{n+1}$. Бралась сетка с 21-й точкой по каждой стороне $(h=0.05)$. $\epsilon = 10^{-6}$.

Результирующий график итераций изображен здесь:

\begin{figure}[h]
\centering
\includegraphics[width=0.7\textwidth]{Seidel.png}
\caption{График численного решения задачи}
\end{figure}

Ошибка решения по отношению к аналитическому решению оказалась равна $err = 3.92 * 10^{-5}$


\newpage
\section{Метод Фурье}

Метод Фурье работает от разложения функции правой части $f_{kl}$ на собственные функции $g_s(l) = g_s(y_l)$ и коэффициенты $\hat f_s(k)$.
Сеточная функция $y_{kl}$ так же разложима по базису $g_s(l) = g_s(y_l)$, но уже с другими коэффициентами, $c_s(k)$.
Стоит отметить, что мы фиксируем значение k, то есть координату $x$, чтобы работать только с одной координатой $y$:

\[ y_{kl} = \sum_{s=1}^{N_2-1}c_s(k)g_s(l) \]
\[ f_{kl} = \sum_{s=1}^{N_2-1} \hat f_s(k) g_s(l) \]

\[ \hat f_s(k) = h \sum_{l=1}^{N_2-1}f_{kl}g_s(l) \]

Однако при подстановке в уравнения схемы для каждой внутренней точки $y_{kl}$, ситуация обратная - чтобы рассматривать коэффициенты при каждой базисной функции по отдельности, теперь мы фиксируем $s$.
В итоге мы получим s систем уравнений по координате $x$:

\[ \frac{c_s(k+1) - 2 c_s(k) + c_s(k-1)}{h^2} - \lambda_s c_s(k) = \hat f_s(k), \ \ k = 1, 2, \dots, N_1-1\]
\[ c_s(0) = c_s(N_1) = 0 \]
\[ N_1 = N_2 = N \]

Нахождение коэффициентов $c_s(k)$ позволяет по собственному базису восстановить внутренние значения сетки.
Важно, помнить что собственные функции и значения:

\[ g_s(l) = \sqrt{\frac{2}{h*N} \sin{\frac{\pi s y_l}{h*N}} \]
\[ \lambda_s = \frac{4}{h^2} \sin^2{\frac{\pi s}{2 N}}, s=1, \dots, N-1 \]

Известны только для задачи с однородными краевыми условиями. 
При неоднородных краевых условиях сеточная функция правой части принимает все ненулевые значение краёв из всех граничных положений шаблона - тогда мы получаем однородные краевые условия с измененной функцией правой части.

Код реализации метода описан в приложении \ref{foir}.

Результирующий график метода Фурье изображен здесь:

\begin{figure}[h]
\centering
\includegraphics[width=0.7\textwidth]{Fourier.png}
\caption{График численного решения задачи методом Фурье}
\end{figure}

Ошибка решения по отношению к аналитическому решению оказалась равна $err = 4.2*10^{-16}$

\newpage

\section{Приложение}

\begin{lstlisting}[language=Python, caption=Реализация явного метода на языке Python, label=seid]
import numpy as np

def mu_x1_0(x2):
    return x2 ** 2

def mu_x2_0(x1):
    return x1 ** 2

def mu_x1_1(x2):
    return 1+ x2 ** 2

def mu_x2_1(x1):
    return 1+ x1 ** 2

N = 20
x, h = np.linspace(0,1,N+1, retstep=True)
u = np.zeros((N+1,N+1))

for i in range(N+1):
    for j in range(N+1):
        if i == 0:
            u[i][j] = mu_x1_0(j * h)
        if j == 0:
            u[i][j] = mu_x2_0(i * h)
        if i == N:
            u[i][j] = mu_x1_1(j * h)
        if j == N:
            u[i][j] = mu_x1_1(i * h)

def f(x,y):
    return 4

error = 1
while error > 10**-6:
    u_0 = u.copy()
    for i in range(1, N):
        for j in range(1, N):
            u[i][j] = 0.25 * (u[i - 1][j] + u_0[i + 1][j] + u[i][j - 1] + u_0[i][j + 1] - h ** 2 * f(h*i, h*j))
    error = np.linalg.norm(u - u_0) * h

\end{lstlisting}

\newpage

\begin{lstlisting}[language=Python, caption=Реализация метода Фурье на языке Python, label=foir]
import numpy as np
from scipy.linalg import solve_banded
import matplotlib.pyplot as plt

def mu_x1_0(x2):
    return x2 ** 2

def mu_x2_0(x1):
    return x1 ** 2

def mu_x1_1(x2):
    return 1+ x2 ** 2

def mu_x2_1(x1):
    return 1+ x1 ** 2

def true_f():
    return 4

def f(i, j, h, N):
    result = true_f()
    if i != 0 and j != 0 and i != N and j != N:
        if i == 1:
            result -= mu_x1_0(j*h)/h**2
        if j == 1:
            result -= mu_x2_0(i*h)/h**2
        if i == N - 1:
            result -= mu_x1_1(j*h)/h**2
        if j == N - 1:
            result -= mu_x2_1(i*h)/h**2
    return result

def gsl(s, N, h, x):
    a = N * h
    return np.sqrt(2 / a) * np. sin(np.pi * s * x / a)

def ls(s, N, h):
    a = N * h
    return 4 / h**2 * np.sin(np.pi * s / (2*N)) ** 2

N=10
x, h = np.linspace(0,1,N+1, retstep=True)
f_arr = np.zeros((N+1,N+1))
for i in range(N+1):
    for j in range(N+1):
        f_arr[i,j] = f(i, j, h, N)

fsk = np.zeros((N-1, N-1))
for k in range(1, N):
    for s in range(1, N):
        fsk[k-1, s-1] = h * np.sum(f_arr[k, 1:-1] * gsl(s, N, h, x)[1:-1])

lambdas = np.zeros(N-1)
for s in range(1,N):
    lambdas[s-1] = ls(s, N, h)


csk = np.zeros((N+1, N-1))
for s in range(1, N):
    A_s = np.zeros((3, N-1))
    A_s[0] = 1/h**2
    A_s[1] = -2/h**2 - lambdas[s-1]
    A_s[2] = 1/h**2
    csk[1:-1, s-1] = solve_banded((1,1), A_s, fsk[:, s-1])

y = np.zeros((N+1, N+1))
for k in range(0, N+1):
    for l in range(0, N+1):
        y[k,l] = np.sum(csk[k,:] * gsl(l, N, h, x)[1:-1])

y[0,:] = mu_x1_0(x)
y[-1,:] = mu_x1_1(x)
y[:,0] = mu_x2_0(x)
y[:,-1] = mu_x2_1(x)
\end{lstlisting}
